\documentclass[11pt,a4paper,sans]{moderncv}        % possible options include font size ('10pt', '11pt' and '12pt'), paper size ('a4paper', 'letterpaper', 'a5paper', 'legalpaper', 'executivepaper' and 'landscape') and font family ('sans' and 'roman')
% moderncv themes
\moderncvstyle{classic}                            % style options are 'casual' (default), 'classic', 'oldstyle' and 'banking'
\moderncvcolor{red}                                % color options 'blue' (default), 'orange', 'green', 'red', 'purple', 'grey' and 'black'
%\renewcommand{\familydefault}{\sfdefault}         % to set the default font; use '\sfdefault' for the default sans serif font, '\rmdefault' for the default roman one, or any tex font name
%\nopagenumbers{}                                  % uncomment to suppress automatic page numbering for CVs longer than one page
\usepackage[utf8]{inputenc}
\inputencoding{latin1}
\inputencoding{utf8}
\usepackage{hyperref}
\hypersetup{colorlinks,urlcolor=blue}
\usepackage[utf8]{inputenc}                    \usepackage[scale=0.75]{geometry}

% personal data
\name{Louie D.}{Corpe}
%\title{Research Statement}
\address{14 Sidney Terrace}{CM23 3TU}{United Kingdom}
\phone[mobile]{+447948343026}                  
\email{ldcorpe@msn.com}                               % optional, remove / comment the line if not wanted
%\homepage{www.johndoe.com}                         
%\extrainfo{additional information}                 
%\photo[64pt][0.42pt]{picture}                       
%\makeatletter
%\renewcommand*{\bibliographyitemlabel}{\@biblabel{\arabic{enumiv}}}
%\makeatother
%\renewcommand*{\bibliographyitemlabel}{[\arabic{enumiv}]}% CONSIDER REPLACING THE ABOVE BY THIS

% bibliography with mutiple entries
%\usepackage{multibib}
%\newcites{book,misc}{{Books},{Others}}
%----------------------------------------------------------------------------------
%            content
%----------------------------------------------------------------------------------
\begin{document}
%\begin{CJK*}{UTF8}{gbsn}                          % to typeset your resume in Chinese using CJK
%-----       resume       ---------------------------------------------------------
\recipient{Department of Physics}{Princeton University\\New Jersey\\United States}
\date{\today}
\opening{Dear Professor Page,}
\closing{Yours faithfully,}
%\enclosure[Attached:]{Curriculum Vitae}          % use an optional argument to use a string other than "Enclosure", or redefine \enclname
%\enclosure[References]{
%\begin{itemize}
%\item \textbf{[1]} blah blah
%\end{itemize}
%}          % use an optional argument to use a string other than "Enclosure", or redefine \enclname
\makelettertitle

I would like to express my interest in applying for a Dicke Fellowship, as recently advertised on \texttt{inspirehep.net}, in order to work with the Princeton Experimental High Energy Physics (HEP) group on the Compact Muon Solenoid (CMS) experiment.  

My PhD at Imperial College London involved the study of the Higgs boson decaying to photons ($H \rightarrow \gamma \gamma$) with data collected at the CMS experiment at the Large Hadron Collider (LHC). I was supervised by Prof Paul Dauncey and Dr Chris Seez, and will be submitting my thesis in April 2017. Over the course of my PhD I have developed a strong background in Higgs physics and statistical analysis techniques. Through my service work on the CMS Electromagnetic Calorimeter (ECAL), I have acquired experience in detector monitoring and calibration which are relevant to your group's work on luminosity measurement. For these reasons, I believe I am well suited to the Princeton HEP group's research interests and could make a valuable contribution to your research programme.

As one of the lead analysers of the $H \rightarrow \gamma \gamma$ decay, I was responsible for signal and background modelling, and statistical interpretation of the data. Using a dataset collected in the first half of 2016, we made a new observation of the Higgs boson, with over 5$\sigma$ significance in the diphoton decay channel alone, and  made measurements of its production rate and the strength of its interaction with other particles.
I have made several contributions to the undertsanding of photons at CMS, including assessing the final energy resolution of photons during the first run of the LHC, and contributing to the monitoring and calibration of the ECAL at the start of the second run. I also estimated the photon identification efficiency of the proposed High Granularity Calorimeter upgrade of the CMS detector for the High-Luminosity LHC (HL-LHC). Finally, I used the our $H \rightarrow \gamma \gamma$ results to extrapolate our analysis to the conditions of the HL-LHC. This was used to inform future physics strategy discussions at the ECFA workshop 2016.

 %Many extensions to the Standard Model (SM) of particles physics imply modifications to the properties and multiplicities Higgs bosons. Were I to obtain the Dicke Fellowship, my research would therefore focus on measurements of the properties of the Higgs boson. Any deviations from the SM predictions could yield strong clues to the nature of new physics, which must exist to explain neutrino oscillations, dark matter and the Hierachy problem. An obvious synergy in the short term would be to use my understanding of photons and my experience analysing the $H \rightarrow \gamma \gamma$ decay to enhance existing efforts in the Princeton group to search for di-Higgs production alongside $b$-quarks and photons. In the longer term, as the LHC accumulates more data and statistical uncertainties diminish, observables which minimize theory dependence will play a more important role. I would seek to help develop and implement such variables to make more stringent tests on the SM predictions in the Higgs sector.
The discovery of the Higgs boson has opened new and exciting research avenues. As a Dicke Fellow, I would seek to use the expertise acquired during my PhD to make precision measurements of the Higgs boson's properties to take advantage of this tremendous opportunity.

I would appreciate an opportunity to discuss my experience and research plans in more detail at interview, and look forward to hearing from you shortly,



\makeletterclosing

\end{document}


\end{document}
