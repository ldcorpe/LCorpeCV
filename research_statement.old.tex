\documentclass[11pt,a4paper,sans]{moderncv}        % possible options include font size ('10pt', '11pt' and '12pt'), paper size ('a4paper', 'letterpaper', 'a5paper', 'legalpaper', 'executivepaper' and 'landscape') and font family ('sans' and 'roman')
% moderncv themes
\moderncvstyle{classic}                            % style options are 'casual' (default), 'classic', 'oldstyle' and 'banking'
\moderncvcolor{black}                                % color options 'blue' (default), 'orange', 'green', 'red', 'purple', 'grey' and 'black'
%\renewcommand{\familydefault}{\sfdefault}         % to set the default font; use '\sfdefault' for the default sans serif font, '\rmdefault' for the default roman one, or any tex font name
%\nopagenumbers{}                                  % uncomment to suppress automatic page numbering for CVs longer than one page
\usepackage[utf8]{inputenc}
\inputencoding{latin1}
\inputencoding{utf8}
\usepackage{hyperref}
\hypersetup{colorlinks,urlcolor=blue}
\usepackage[utf8]{inputenc}                    \usepackage[scale=0.75]{geometry}

% personal data
\name{Louie D.}{Corpe}
%\title{Research Statement}
\address{Research Statement}{Dicke Fellowship - Princeton University}{}
%\phone[mobile]{+447948343026}                  
%\email{ldcorpe@msn.com}                               % optional, remove / comment the line if not wanted
%\homepage{www.johndoe.com}                         
%\extrainfo{additional information}                 
%\photo[64pt][0.42pt]{picture}                       
%\makeatletter
%\renewcommand*{\bibliographyitemlabel}{\@biblabel{\arabic{enumiv}}}
%\makeatother
%\renewcommand*{\bibliographyitemlabel}{[\arabic{enumiv}]}% CONSIDER REPLACING THE ABOVE BY THIS

% bibliography with mutiple entries
%\usepackage{multibib}
%\newcites{book,misc}{{Books},{Others}}
%----------------------------------------------------------------------------------
%            content
%----------------------------------------------------------------------------------
\begin{document}
%\begin{CJK*}{UTF8}{gbsn}                          % to typeset your resume in Chinese using CJK
%-----       resume       ---------------------------------------------------------
\recipient{}{}
\date{\today}
\opening{}
\closing{}
\enclosure[]{}          % use an optional argument to use a string other than "Enclosure", or redefine \enclname
\makelettertitle

The discovery of the Higgs boson in 2012 marked the end of the final chapter in the search for the particles predicted by the Standard Model (SM) of particle physics. The SM has been one of the most successful theories of modern physics in terms of its predictive power and agreement with experiment, but it is clearly incomplete. It does not account for observed oscillations of neutrino flavour or leave room for the existence of dark matter. Furthermore, the observed mass of the Higgs boson at 125~GeV raises the issue of the fine-tuning problem. Many extensions to the SM have been proposed which could provide answers to these open questions. In some models the Higgs boson is in fact a composite particle. In others it is part of a wider family of Higgs bosons. Other theories suggest that Higgs boson could be a 'portal' to new physics, as it could interact with unknown particles with which regular matter cannot. In all cases, variations in the measurable properties of the Higgs boson would be predicted. The discovery of the Higgs boson has therefore opened a new avenue with which to study the fundamental questions about our universe. This avenue will be exploited for the very first time in the coming decade. The relvant tools will be: precise measurements of the rate of production of the Higgs boson; the strength of its interaction with the particles which we know about; searches for additional Higgs bosons. Any deviation from the SM predictions could be a sign of new physics. As Dicke fellow I would seek to take up this outstanding opportunity.

The work I have carried out during my Ph.D. has been focussed around the $H \rightarrow \gamma \gamma$ decay. In 2015, after an upgrade period, the LHC started its second data-taking run, raising its collision energy from 8~TeV to 13~TeV. An important task for the higher energy physics (HEP) community was to 'rediscover' the Higgs boson. My role in the CMS $H \rightarrow \gamma \gamma$ was to undertake the signal and background modelling, and perform the statistical interpretation of the data. Using the data collected in the first half of 2016, we were able to claim a new observation of the Higgs boson, with over 5$\sigma$ significance in the diphoton decay channel alone~[1]. The helped to confirm beyond any doubt the existence of the particle. The LHC is accumulating data at an incredible rate, and future upgrades will accelerate data-taking even more. The measurements of the Higgs boson's properties so far are all consistent with the expectations of the SM, but with large uncertainties. For example, our latest measurement of the strength of the Higgs boson signal in the $H \rightarrow \gamma \gamma$ decay was 95\% of that expected by the SM, but with 20\% uncertainty. Some of the extensions of the SM would predict subtlety larger or smaller values, and there is plenty of room for such variations to hide within the current uncertainties. Recent extrapolations which I carried out showed that reducing these uncertainties to the 6-10\% range could be achievable in the next five years, and down to the 4-8\% range by the end of the LHC programme, in the diphoton decay channel alone. Such improvement would put stringent constraints on the existence and nature of physics beyond the SM.

Were I to obtain the Dicke Fellowship, I would seek to focus some of my research on measurements of the properties of the Higgs boson, in an effort to discover clues to the nature of beyond the SM physics. As the LHC accumulates more data and statistical uncertainties in the main Higgs decay channels diminish, observables which minimize theory dependence, such as differential cross sections, will play an important role. Furthermore, testing the predictions of the SM will require the development and implementation of new observables which can be easily interpreted by the theory community, and compared to results from other experiments. The simplified cross-sections scheme is a good example of this. Efforts are already underway to begin using a basic version of the scheme in both CMS and ATLAS, although successful adoption will require synchronisation between the two experiments. My experience in the statistical analysis of $H \rightarrow \gamma \gamma$ results and my responsibilities as the CMS Higgs Combination group (HCG)  $H \rightarrow \gamma \gamma$ contact-person place me in an ideal situation to assist with the development and implementation of the scheme. Although this is not an area which the Princeton HEP group is currently involved in, I believe it is a high-impact and high-visibility activity which would benefit the group, and I would therefore seek to work in this domain if selected to be a Dicke Fellow. 

An obvious synergy would be to use my experience in analysing the $H \rightarrow \gamma \gamma$ decay alongside existing efforts in the Princeton group to search for di-Higgs production alongside $b$-quarks ($X \rightarrow HH \rightarrow \gamma \gamma bb$). Indeed, both analyses use the \texttt{FLASHgg} software framework which I helped to design and build, and the signal and background modelling techniques are closely related. As contact-person for the HCG, I was responsible for checking that the inputs to the $X \rightarrow HH \rightarrow \gamma \gamma bb$ statistical analysis were consistent, so I am already familiar with this analysis. The analysis of $X \rightarrow HH \rightarrow \gamma \gamma bb$ is projected to remain statistically limited until the end of the LHC programme, so future developments would rely on improvements to the efficiency$\times$acceptance of the analysis. One way to do this, at least for the non-resonant component of the analysis, would be to consider other production modes for the Higgs boson. At present only the gluon-fusion mode is taken into account, whereas at least 10\% of Higgs bosons are produced via other methods.  

Finally, I would appreciate the opportunity to broaden my understanding of collider physics by taking part in elements of service work. The Princeton HEP group's involvement in the measurement of the luminosity of the LHC beam, and therefore the amount of data collected by CMS, would represent a fascinating challenge, and one which will become very important as the integrated luminosity delivered by the LHC becomes large. Indeed, the uncertainty on the luminosity measurement at CMS, at present, is of the order of 3-6\%. For analyses which are statistically dominated, this can be one of the major sources of uncertainty, and therefore I would be very interested in helping to drive down the size of this uncertainty for the benefit of the wider CMS collaboration.

\makeletterclosing

\end{document}


\end{document}
