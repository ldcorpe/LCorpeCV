\documentclass[12pt,a4paper]{moderncv}

\moderncvstyle{classic}
\moderncvcolor{black}

\usepackage[scale=0.85]{geometry}
\usepackage[english]{babel}

\firstname{Patrick}
\familyname{Dunne}

\address{Flat 2, 45A Kingston Road,}{ London, SW19 1JW}
\mobile{(+44) 7919 405 613}
\email{dunnepatrickj@gmail.com}

\newcommand\myitem{\item \normalsize }
\newcommand\mysection[1]{\section{#1}}

\renewcommand*{\cventry}[7][.25em]{%
  \cvitem[#1]{#2}{%
    {\bfseries#3}%
    \ifthenelse{\equal{#4}{}}{}{, {\slshape#4}}% I changed this line (with comma) ...
      %\ifthenelse{\equal{#4}{}}{}{ {\slshape#4}}% ... into this one (without comma).
      \ifthenelse{\equal{#5}{}}{}{, #5}%
      \ifthenelse{\equal{#6}{}}{}{, #6}%
      \strut%
      \ifx&#7&%
      \else{\newline{}\begin{minipage}[t]{\linewidth}\small#7\end{minipage}}\fi}}

      %\namefont \titlefont \addressfont \mysectionfont \subsectionfont \hintfont \quotefont

      \begin{document}

      \makecvtitle

      %\vspace{-1.5cm}

      \mysection{Research Experience}
      \cventry{2012-present}{PhD Student}{Imperial College}{Expect to submit early in 2016}{}{
        \begin{itemize}
        \myitem Carried out searches for invisibly decaying Higgs bosons with the CMS detector focussing on the most powerful VBF production channel
          \vspace{.1cm}
        \myitem Led the combination of all CMS direct searches for invisible Higgs boson decays:
          \begin{itemize}
        \myitem Carried out the statistical combination of the searches, studies of correlations between channels and set limits on the Higgs boson invisible branching fraction
          \myitem Made public in CMS physics analysis summary HIG-15-012
          \end{itemize}
        \vspace{.1cm}
        \myitem Led the search for invisibly decaying Higgs bosons in the VBF channel using parked CMS data from LHC run I
          \begin{itemize}
        \myitem Responsible for trigger efficiency measurements, background estimation, systematic uncertainty studies and the setting of limits on the Higgs boson invisible branching fraction
          \myitem Made public in CMS physics analysis summary HIG-14-038
          \end{itemize}
        \vspace{.1cm}
        \myitem One of the lead analysers in the search for invisibly decaying Higgs bosons in the VBF channel using prompt CMS data from LHC run I
          \begin{itemize}
        \myitem Performed background estimation and systematic uncertainty studies and set limits on the Higgs boson invisible branching fraction
          \myitem Led the combination of this result with CMS searches in other channels, performing studies of correlations between the channels and limit setting
          \myitem Made public in a paper selected for the cover of the European Physics Journal C and CMS physics analysis summary HIG-13-013
          \end{itemize}
        \vspace{.1cm}
        \myitem Currently interpreting the results of these searches as limits on Higgs portal, two Higgs doublet and effective field theory dark matter models and studying data from LHC run II
          \vspace{.1cm}
        \myitem Co-founded a student seminar club to improve the presentation skills of students and inform the group of our research activities
          \end{itemize}
        \vspace{.1cm}
      }
\cventry{2013}{Long Term Attachment}{CERN}{}{}{
  \begin{itemize}
  \myitem Based at CERN for four months as part of my PhD research
    \myitem Carried out detector operations shifts on the CMS detector
    \end{itemize}
  \vspace{.1cm}
}

\cventry{2012}{Masters Project}{University of Oxford}{}{}{
  \begin{itemize}
  \myitem Investigated the `Qjets' algorithm for stochastic jet clustering as part of the ATLAS collaboration
    \myitem Identified improvements in the resolution of objects within jets with high transverse momentum
    \end{itemize}
  \vspace{.1cm}
}

\cventry{2005-2011}{Summer Projects}{LLNL, Oxford, CERN, MIT}{}{}{
  \begin{itemize}
  \myitem 2011 - Lawrence Livermore National Laboratory (LLNL), California: Performed studies of metals at high pressure using the Jupiter Laser Facility
    \myitem 2010 - University of Oxford: Analysed data from the CDF experiment searching for non-standard model couplings in VBF produced W bosons
    \myitem 2007 - CERN: Assisted in assembling a frequency scanning interferometry system for monitoring of the ATLAS semiconductor tracker
    \myitem 2005 - MIT: Studied algorithms for robot control in the rapid prototyping group
    \end{itemize}
}

\mysection{Education}
\cventry{2008-2012}{MPhys Physics}{University of Oxford}{First class honours degree}{}{}
\cventry{2005-2008}{5 A levels, 12 GCSEs (2 short course)}{Sutton Grammar School}{}{}{
  \begin{itemize}
  \myitem Grade A in physics, maths, further maths, electronics and chemistry A levels
    \myitem Advanced extension awards in physics (distinction) and maths (merit)
    \end{itemize}
}

\mysection{Academic Awards}
\cventry{2014}{Poster Prize}{Imperial College Physics Department}{}{}{
  \begin{itemize}
  \myitem Prize at the postgraduate summer research symposium
    \end{itemize}
  \vspace{.04cm}
}
\cventry{2014}{Poster Prize}{Imperial College Graduate School}{}{}{
  \begin{itemize}
  \myitem Won second prize in the college wide graduate school poster competition
    \end{itemize}
  \vspace{.04cm}
}
\cventry{2013}{Poster Prize}{STFC}{}{}{
  \begin{itemize}
  \myitem Awarded for STFC high energy physics summer school poster competition
    \end{itemize}
  \vspace{.04cm}
}
\cventry{2012}{Peter Fisher Prize}{Trinity College Oxford}{}{}{
  \begin{itemize}
  \myitem Prize awarded to the student in college with the best finals results in physics
    \end{itemize}
  \vspace{.04cm}
}
\cventry{2012}{Mitchell Scholarship for Outstanding Students}{Trinity College Oxford}{}{}{
  \begin{itemize}
  \myitem Scholarship awarded to allow promising students to undertake research projects
    \myitem Funds won used to participate in studies of metals at very high pressures using the Jupiter Laser Facility at Lawrence Livermore National Laboratory
    \end{itemize}
  \vspace{.04cm}
}
\cventry{2010-2012}{Millard Scholarship}{Trinity College Oxford}{}{}{
  \begin{itemize}
  \myitem Awarded for continued excellent performance in examinations
    \end{itemize}
  \vspace{.04cm}
}
\cventry{2010}{Gibbs Prize for Public Speaking}{University of Oxford Physics Department}{}{}{
  \begin{itemize}
  \myitem Best talk in department-wide physics speaking competition with 170 entries
    \end{itemize}
  \vspace{.04cm}
}
\cventry{2010}{Examiners' Commendation}{Oxford University Physics Department}{}{}{
  \begin{itemize}
  \myitem Awarded for outstanding performance in second year practical course
    \end{itemize}
  \vspace{.04cm}
}
\cventry{2009}{Millard Exhibition}{Trinity College Oxford}{}{}{
  \begin{itemize}
  \myitem Awarded for performance in preliminary examinations
    \end{itemize}
  \vspace{.04cm}
}
\cventry{2008}{Neate Physics Prize}{Sutton Grammar School}{}{}{
  \begin{itemize}
  \myitem Awarded to best physics student in the school
    \end{itemize}
  \vspace{.04cm}
}



\mysection{Teaching Experience}
\cventry{2014-present}{Masters Student Supervision}{Imperial College}{}{}{
  \begin{itemize}
  \myitem Supervised/supervising five students undertaking projects on searches for VBF produced invisible Higgs boson decays and their interpretations
    %   \myitem Padraic Calpin (Imperial College), Searches for VBF produced invisible Higgs
    %   \myitem Achilleas Fragkoulis (Imperial College), Searches for VBF produced invisible Higgs
    %   \myitem Miha Zgubic (Imperial College), Interpretations of searches for invisible Higgs
    \end{itemize}
  \vspace{.05cm}
}
\cventry{2013-present}{Undergraduate Lab Demonstration}{Imperial College}{}{}{
  \begin{itemize}
  \myitem Demonstrated in second year radioactivity lab
    \myitem Voted best lab demonstrator by students in AC 2013/14 and 2014/15
    \myitem Responsible for interviewing students and marking their work
    \end{itemize}
  \vspace{.05cm}
}


\mysection{Outreach}
\cventry{2015}{Searching for the Higgs boson at the LHC}{Sutton Grammar School}{}{}{}
\vspace{.05cm}
\cventry{2014}{Royal Society Summer Exhibition}{London}{}{}{}
\vspace{.05cm}
\cventry{2014}{High Energy Physics Masterclass}{Imperial College, London}{}{}{}
\vspace{.05cm}
\cventry{2013}{Tour Guide}{CMS detector, CERN}{}{}{}
\vspace{.05cm}
\cventry{2013}{Bang Fair}{ExCel Centre, London}{}{}{}
\vspace{.05cm}
\cventry{2013}{High Energy Physics Masterclass}{Imperial College, London}{}{}{}




\mysection{Memberships and Collaborations}
\cventry{2012-present}{CMS Collaboration}{}{}{}{}
\vspace{.05cm}
\cventry{2008-present}{Institute of Physics}{}{}{}{}
\vspace{.05cm}
\cventry{2012}{ATLAS Collaboration}{}{}{}{}

\mysection{Talks}
\cventry{2015}{CMS Higgs Group}{Approval and pre-approval talks for analysis HIG-15-012}{`A combination of searches for the invisible decays of the Higgs boson using the CMS detector'}{}{}
\vspace{.05cm}
\cventry{2015}{IOP HEP Group Conference}{Manchester}{`Searches for invisible decays of the Higgs boson with the CMS detector'}{}{}
\vspace{.05cm}
\cventry{2015}{CMS UK Collaboration Meeting}{}{`VBF Higgs to Invisible - Towards Run II'}{}{}
\vspace{.05cm}
\cventry{2014}{CMS Higgs Group}{Pre-approval talk for analysis HIG-14-038}{`Search for invisible decays of Higgs bosons in the vector boson fusion production mode'}{}{}
\vspace{.05cm}
\cventry{2014}{PANIC Conference}{Hamburg}{`Searches for invisible decay modes of the Higgs boson with the CMS detector'}{}{}
\vspace{.05cm}
\cventry{2014}{CMS Higgs Group}{Approval talk for analysis HIG-13-030}{`Search for invisible decays of Higgs bosons in the vector boson fusion and associated ZH production modes'}{}{}
\vspace{.05cm}
\cventry{2014}{CMS UK Collaboration Meeting}{}{`Higgs to invisible analyses at CMS'}{}{}
\vspace{.05cm}

\mysection{Publications}
\normalsize Co-author on 154 citable papers as part of the CMS collaboration (inSPIRE HEP, January 2016) and have an $h_{hep}$ index of 34. Selected papers with substantial contributions are given below: %!! make sure up to date

\medskip

\cventry{2015}{CMS Collaboration}{`A combination of searches for the invisible decays of the Higgs boson using the CMS detector'}{CMS Physics Analysis Summary - HIG-15-012}{}{
  \begin{itemize}
  \myitem Was the main analysis contact for the CMS review process
    \myitem Led the analysis and performed the statistical combination of the separate searches, studies of correlations between channels and limit setting
    \end{itemize}
  \vspace{.1cm}
}
\cventry{2015}{CMS Collaboration}{`Search for invisible decays of Higgs bosons in the vector boson fusion production mode'}{CMS Physics Analysis Summary - HIG-14-038}{}{
  \begin{itemize}
  \myitem Was the main analysis contact for the CMS review process
    \myitem Led the analysis and was responsible for trigger efficiency measurement, background estimation, systematic uncertainty studies and setting limits on the Higgs boson invisible branching fraction
    \end{itemize}
  \vspace{.1cm}
}
\cventry{2014}{CMS Collaboration}{`Search for invisible decays of Higgs bosons in the vector boson fusion and associated ZH production modes'}{Eur. Phys. J. C 74 (2014) 2980}{}{
  \begin{itemize}
  \myitem One of the lead analysers responsible for background estimation, systematic uncertainty studies and the setting of limits on the Higgs boson invisible branching fraction in the most powerful VBF channel
    \myitem Led the statistical combination of the separate searches and studies of correlations between channels
    \myitem This paper has received 134 citations (inSPIRE HEP, January 2016)
    \end{itemize}
  \vspace{.1cm}
}
\cventry{2013}{CMS Collaboration}{`Search for invisible decays of Higgs bosons in the VBF channel'}{CMS Physics Analysis Summary - HIG-13-013}{}{
  \begin{itemize}
  \myitem One of the lead analysers responsible for background estimation and systematic uncertainty studies
    \myitem First limit on the Higgs boson's invisible branching fraction in the most sensitive VBF channel
    \end{itemize}
  \vspace{.1cm}
}

\mysection{Further Information}
\cventry{}{Computing}{}{}{}{
  \begin{itemize}
  \myitem Experienced in C++, C, python and the ROOT analysis framework
    \myitem Use the MadGraph and Delphes packages for Monte Carlo event generation and detector simulation
    \end{itemize}
  \vspace{.1cm}
}
\cventry{}{Languages}{}{}{}{
  \begin{itemize}
  \myitem English (native), French and German (limited working proficiency)
    \end{itemize}
  \vspace{.1cm}
}
\cventry{}{Positions of Responsibility}{}{}{}{}
\cventry{2014-present}{\textmd{Member of Imperial College high energy physics student seminar organising committee}}{}{}{}{}
\cventry{2010-2013}{\textmd{Youth representative to national council of the Scout Association}}{}{}{}{}
\cventry{2010-2012}{\textmd{Returning officer and webmaster for Trinity College Junior and Middle Common Rooms}}{}{}{}{}


\mysection{References}
\cventry{}{Prof. Gavin Davies}{Professor of High Energy Physics}{Imperial College}{g.j.davies@imperial.ac.uk}{}
%%\cventry{}{Dr. David Colling}{Reader in high energy physics}{Imperial College}{}{d.colling@imperial.ac.uk}
\cventry{}{Dr. James Brooke}{Research associate}{University of Bristol}{}{james.brooke@bristol.ac.uk}
\cventry{}{Dr. Marco Pieri}{Researcher}{University of California San Diego}{}{marco.pieri@cern.ch}


\end{document}
