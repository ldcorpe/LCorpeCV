\documentclass[11pt,a4paper,sans]{moderncv}        % possible options include font size ('10pt', '11pt' and '12pt'), paper size ('a4paper', 'letterpaper', 'a5paper', 'legalpaper', 'executivepaper' and 'landscape') and font family ('sans' and 'roman')
% moderncv themes
\moderncvstyle{classic}                            % style options are 'casual' (default), 'classic', 'oldstyle' and 'banking'
\moderncvcolor{black}                                % color options 'blue' (default), 'orange', 'green', 'red', 'purple', 'grey' and 'black'
%\renewcommand{\familydefault}{\sfdefault}         % to set the default font; use '\sfdefault' for the default sans serif font, '\rmdefault' for the default roman one, or any tex font name
%\nopagenumbers{}                                  % uncomment to suppress automatic page numbering for CVs longer than one page
\usepackage[utf8]{inputenc}
\inputencoding{latin1}
\inputencoding{utf8}
\usepackage{hyperref}
\hypersetup{colorlinks,urlcolor=blue}
\usepackage[utf8]{inputenc}                    \usepackage[scale=0.75]{geometry}

% personal data
\name{Louie D.}{Corpe}
%\title{Research Statement}
\address{Research Statement}{Dicke Fellowship - Princeton University}{}
%\phone[mobile]{+447948343026}                  
%\email{ldcorpe@msn.com}                               % optional, remove / comment the line if not wanted
%\homepage{www.johndoe.com}                         
%\extrainfo{additional information}                 
%\photo[64pt][0.42pt]{picture}                       
%\makeatletter
%\renewcommand*{\bibliographyitemlabel}{\@biblabel{\arabic{enumiv}}}
%\makeatother
%\renewcommand*{\bibliographyitemlabel}{[\arabic{enumiv}]}% CONSIDER REPLACING THE ABOVE BY THIS

% bibliography with mutiple entries
%\usepackage{multibib}
%\newcites{book,misc}{{Books},{Others}}
%----------------------------------------------------------------------------------
%            content
%----------------------------------------------------------------------------------
\begin{document}
%\begin{CJK*}{UTF8}{gbsn}                          % to typeset your resume in Chinese using CJK
%-----       resume       ---------------------------------------------------------
\recipient{}{}
\date{\today}
\opening{}
\closing{}
%\enclosure[]{}          % use an optional argument to use a string other than "Enclosure", or redefine \enclname
\makelettertitle

In 2015 the Large Hadron Collider (LHC) started its second data-taking run, raising its collision energy from 8~TeV to 13~TeV. An important task for the higher energy physics (HEP) community was to 'rediscover' the Higgs boson, which was first observed in 2012. My role in the CMS analysis of the  $H \rightarrow \gamma \gamma$ decay was to undertake the modelling of the signal and background, and perform the statistical interpretation of the data. Using the dataset collected in the first half of 2016, we were able to present a new observation of the Higgs boson, with over 5$\sigma$ significance in the diphoton decay channel alone~\cite{PAS-HIG-16-020}. This result helped to confirm beyond any doubt the existence of the particle, and cemented the SM as one of the most successful theories of modern physics.


Despite its successes, the SM is not a satisfactory theory to describe our universe. It does not account for oscillations of neutrino flavor or leave room for the existence of dark matter. Furthermore, the observed mass of the Higgs boson raises the issue of the fine-tuning problem. Many extensions to the SM have been proposed which could provide answers to these open questions. In some models the Higgs boson is in fact a composite particle. In others it is part of a wider family of Higgs bosons. Other theories suggest that the Higgs boson could be a 'portal' to new physics, as it could interact with unknown particles with which regular matter cannot. In all cases, variations in the measurable properties of the Higgs boson are predicted. The discovery of the Higgs boson has therefore opened a new avenue with which to study the fundamental questions about our universe. This avenue will be exploited for the very first time in the coming decade, as the field of Higgs physics pivots from the discovery era into the precision measurement era. Searches for additional Higgs bosons, as well as precise measurements of the Higgs boson's productions rates and the strength of its interactions with SM particles, will both be important to perform. Any deviation from the SM predictions could be a sign of new physics. As Dicke fellow I would seek to take advantage of the outstanding opportunities in both of these domains.

At present, all measurements of Higgs boson properties are consistent with SM expectations, but with large uncertainties. For example, CMS's latest measurement of the Higgs boson production rate in the $H \rightarrow \gamma \gamma$ decay was 95\% of the SM expectation, but with 20\% uncertainty. Evidently, there is plenty of room for discrepancies with the SM to hide within the current uncertainties. Recent extrapolations showed that reducing these uncertainties to the 6-10\% range could be achievable in the next five years, and down to the 4-8\% range by the end of the LHC programme, in the diphoton decay channel alone~\cite{CMS-DP-2016-064}. Such improvements will put stringent constraints on the nature of physics beyond the SM, particularly when combined with results from other decay channels. My experience in the analysis of $H \rightarrow \gamma \gamma$ results and my responsibilities as the CMS Higgs Combination Group (HCG)  $H \rightarrow \gamma \gamma$ contact-person put me in an ideal position to assist with such measurements and combinations. Of particular interest is the rate at which Higgs bosons are produced in association with top quarks, since this is sensitive to the existence of additional top-like particles. Although this is not an area which the Princeton HEP group is currently involved in, I believe it is a high-impact and high-visibility activity which would benefit the group, and I would make the case to continue working in this domain if selected to be a Dicke Fellow. 

Regarding searches for additional Higgs bosons, the Princeton HEP group is already involved in the study of the $ \gamma \gamma bb$ final state, which is sensitive to the existence of extra Higgs-like particles decaying to two Higgs bosons ($X \rightarrow HH \rightarrow \gamma \gamma bb$). An obvious synergy would be to use my experience with the $H \rightarrow \gamma \gamma$ decay alongside existing efforts in the Princeton group to refine this analysis. Indeed, both now use the \texttt{FLASHgg} software framework which I helped to design and build, and the signal and background modelling techniques are closely related. The analysis is projected to remain statistically limited until the end of the LHC programme, so future developments would rely on improvements to its signal efficiency and overall sensitivity. At present only the gluon fusion Higgs production mode is taken into account in the non-resonant component of the analysis. At least 10\% of Higgs bosons are produced via other methods. Hence, one possible development  would be to consider other production modes. Working on the $X \rightarrow HH \rightarrow \gamma \gamma bb$ analysis would form another aspect of my research as Dicke Fellow.

Finally, I would appreciate the opportunity to broaden my understanding of collider physics by taking part in elements of service work. The Princeton HEP group's involvement in the measurement of the luminosity of the LHC beam, and therefore the amount of data collected by CMS, would present a fascinating new challenge, and would be an excellent way for me to contribute to the day-to-day running of the CMS experiment. During the first data-taking of 2015 I was involved in the calibration of the CMS Electromagnetic Calorimeter, monitoring the stability of the data collected and validating new calibrations as they arrived. The challenges posed by this type of work would not be dissimilar to those posed by the monitoring and calibration of the luminosity measurement system, and I would be keen to use my experience to help with this process.  %Accurate measurement of luminosity will become even more important as the integrated luminosity delivered by the LHC becomes large. Indeed, the uncertainty on the luminosity measurement at CMS, at present, is of the order of 6\%. For analyses which are statistically dominated, this can be one of the major sources of uncertainty, and therefore I would be very interested in helping to drive down the size of this uncertainty for the benefit of the wider CMS collaboration. 

To sum up, the field of Higgs physics today finds itself in a new and exciting situation. With a dataset expected to grow by a factor of 100 over the course of the LHC programme, precise measurements of the properties of the Higgs boson, as well as searches for new Higgs-like particles, will either put stringent limits on any extensions to the SM or reveal clues to the nature of new physics. If I were to become a Dicke Fellow, I would seek to pursue both aspects of this new research avenue, both by continuing my existing research on precision measurements associated with Higgs boson decays, and by enhancing the effort in the search for additional Higgs-like particles decaying into a $\gamma\gamma bb$ final state. In conjunction with this, I would seek to expand my technical expertise by taking part in the Princeton HEP group's activities in luminosity measurement. 

\makeletterclosing

\vspace{0.42cm}
% Publications from a BibTeX file without multibib
%  for numerical labels: 
\renewcommand{\bibliographyitemlabel}{\@biblabel{\arabic{enumiv}}}% CONSIDER MERGING WITH PREAMBLE PART
%  to redefine the heading string ("Publications"): 
\renewcommand{\refname}{References}
%\nocite{*}
\bibliographystyle{lucas_unsrt}
\bibliography{publications}                        % 'publications' is the name of a BibTeX file

\end{document}
